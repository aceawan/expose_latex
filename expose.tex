\documentclass{beamer}

\mode<presentation>
{
  \usetheme{Warsaw}      % or try Darmstadt, Madrid, Warsaw, ...
  \usecolortheme{default} % or try albatross, beaver, crane, ...
  \usefonttheme{default}  % or try serif, structurebold, ...
  \setbeamertemplate{navigation symbols}{}
  \setbeamertemplate{caption}[numbered]
}

\usepackage[french]{babel}
\usepackage[utf8x]{inputenc}
\usepackage{metalogo}
\usepackage{harmony}
\usepackage{minted}
\usepackage{tikz}
\usepackage{listings}
\usepackage{xcolor}
\usepackage[version=3]{mhchem}

\lstset
{
    language=[LaTeX]TeX,
    breaklines=true,
    basicstyle=\tt\scriptsize,
    keywordstyle=\color{blue},
    identifierstyle=\color{magenta}
}

\title{\LaTeX}
\author{Quentin Ladeveze -- François Lallemand}
\date{\today}

\begin{document}

\begin{frame}
  \titlepage
\end{frame}

\begin{frame}{Qu'est ce que c'est ?}

\begin{itemize}
    \item \TeX : Un logiciel de composition de documents créé en 1977.
    \item Très utilisé dans le monde académique.
    \item Plusieurs variantes modernes: \LaTeX, \XeTeX, \LuaTeX avec chacun leurs avantages (formules, classes  de document, UTF-8)
\end{itemize}

\end{frame}

\begin{frame}[fragile]{Les possibilités}

\begin{itemize}
    \item Les formules mathématiques : $\sum_{n=1}^{+\infty}\frac{1}{n^2}=\frac{\pi^2}{6}$
    \item Les formules chimiques :\ce{A <=> B ^ + C v}
    \item La musique : \ZwPa, \Halb, \AAcht
\end{itemize}

\vfill

\begin{minipage}[t]{0.48\linewidth}
    La coloration syntaxique
    \fontsize{8pt}{8pt}\selectfont
    \hfill
    \vfill
    \begin{minted}[mathescape,
        linenos,
        numbersep=5pt,
        gobble=2,
        frame=lines,
        framesep=2mm]{java}
    // Intégration de formules : $\pi=\lim_{n\to\infty}\frac{P_n}{d}$
    void test(int a){
        System.out.println("valeur : ", a);
    }
    \end{minted}
    \vfill
    \hfill
\end{minipage}
\hfill
\begin{minipage}[t]{0.48\linewidth}
    \hspace{60pt}
    Les graphes
    \vspace{40pt}
    \hfill
    \begin{tikzpicture}
    \node (P) at (0,0) {Paris};
    \node (S) at (1.6, -1) {Strasbourg};
    \node (L) at (1,1) {Lille};
    \draw[->,>=latex] (P) to[bend right] (L);
    \draw[->,>=latex] (L) to[bend left] (S);
    \draw[->,>=latex] (S) to[bend left] (P);
    \end{tikzpicture}
    \hfill
    \vfill
\end{minipage}

\end{frame}

\begin{frame}{Les possibilités}
    \begin{itemize}
        \item \LaTeX \hspace{2pt}gère une grande variété de documents : carte de visite, présentation, mémoire, rapport et beaucoup d'autres.
        \item Possibilité d'exporter dans de nombreux formats : pdf, epub, html+css, dvi
        \item On peut créer des documents \LaTeX \hspace{2pt}avec n'importe quel éditeur de texte, ou avec un IDE prévu à cet effet. On peut créer un documents de dix mille pages sans problèmes.
    \end{itemize}
\end{frame}

\begin{frame}{Le code}
    \begin{itemize}
        \item On cite souvent \TeX \hspace{2pt}comme étant le programme le moins buggé du monde.
        \item Créé en 1977 et très peu modifié depuis.
        \item Les variantes de \TeX \hspace{2pt}sont justes des ensembles de scripts pour faciliter son utilisation.
    \end{itemize}
\end{frame}

\begin{frame}[fragile]{Exemple}

Cet exemple de \LaTeX permet de réaliser le graphe entre Paris, Strasbourg et Lille que l'on a vu dans un des transparents précédents.
\vfill
\begin{lstlisting}
    \documentclass{article}
    \usepackage{tikz}
    \begin{document}
        \begin{tikzpicture}
            \node (P) at (0,0) {Paris};
            \node (S) at (1.6, -1) {Strasbourg};
            \node (L) at (1,1) {Lille};
            \draw[->,>=latex] (P) to[bend right] (L);
            \draw[->,>=latex] (L) to[bend left] (S);
            \draw[->,>=latex] (S) to[bend left] (P);
        \end{tikzpicture}
    \end{document}
\end{lstlisting}
\vfill
\end{frame}

\begin{frame}{Les désavantages}
    \begin{itemize}
        \item La courbe d'apprentissage peut être un peu rude
        \item Quand on modifie son document, on ne voit pas immédiatement ces modifications.
        \item Partager un document \textit{modifiable} peut être un peu compliqué si notre interlocuteur ne connait pas \LaTeX (cf. point 1)
    \end{itemize}
\end{frame}

\begin{frame}{Conclusion}
    \begin{itemize}
        \item \TeX et ses variantes sont un ensemble de logiciels puissants et complet qui permettent de réaliser n'importe quel document.
        \item Ils sont également flexibles que ce soit dans la façon de les écrire ou dans les sorties qu'ils produisent.
        \item Apprendre à utiliser \LaTeX est investissement qui peut paraitre lourd au premier abord mais qui nous affranchit de Word, PowerPoint et de tout un tas de saloperies pour toujours.
    \end{itemize}
\end{frame}
\end{document}

